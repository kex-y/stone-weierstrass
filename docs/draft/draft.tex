% Options for packages loaded elsewhere
\PassOptionsToPackage{unicode}{hyperref}
\PassOptionsToPackage{hyphens}{url}
\PassOptionsToPackage{dvipsnames,svgnames*,x11names*}{xcolor}
%
\documentclass[
]{article}
\usepackage{lmodern}
\usepackage{amssymb,amsmath}
\usepackage{ifxetex,ifluatex}
\ifnum 0\ifxetex 1\fi\ifluatex 1\fi=0 % if pdftex
  \usepackage[T1]{fontenc}
  \usepackage[utf8]{inputenc}
  \usepackage{textcomp} % provide euro and other symbols
\else % if luatex or xetex
  \usepackage{unicode-math}
  \defaultfontfeatures{Scale=MatchLowercase}
  \defaultfontfeatures[\rmfamily]{Ligatures=TeX,Scale=1}
\fi
% Use upquote if available, for straight quotes in verbatim environments
\IfFileExists{upquote.sty}{\usepackage{upquote}}{}
\IfFileExists{microtype.sty}{% use microtype if available
  \usepackage[]{microtype}
  \UseMicrotypeSet[protrusion]{basicmath} % disable protrusion for tt fonts
}{}
\makeatletter
\@ifundefined{KOMAClassName}{% if non-KOMA class
  \IfFileExists{parskip.sty}{%
    \usepackage{parskip}
  }{% else
    \setlength{\parindent}{0pt}
    \setlength{\parskip}{6pt plus 2pt minus 1pt}}
}{% if KOMA class
  \KOMAoptions{parskip=half}}
\makeatother
\usepackage{xcolor}
\IfFileExists{xurl.sty}{\usepackage{xurl}}{} % add URL line breaks if available
\IfFileExists{bookmark.sty}{\usepackage{bookmark}}{\usepackage{hyperref}}
\hypersetup{
  pdftitle={Poster Draft},
  pdfauthor={Kexing Ying},
  colorlinks=true,
  linkcolor=Maroon,
  filecolor=Maroon,
  citecolor=Blue,
  urlcolor=red,
  pdfcreator={LaTeX via pandoc}}
\urlstyle{same} % disable monospaced font for URLs
\usepackage[margin = 1.5in]{geometry}
\usepackage{graphicx}
\makeatletter
\def\maxwidth{\ifdim\Gin@nat@width>\linewidth\linewidth\else\Gin@nat@width\fi}
\def\maxheight{\ifdim\Gin@nat@height>\textheight\textheight\else\Gin@nat@height\fi}
\makeatother
% Scale images if necessary, so that they will not overflow the page
% margins by default, and it is still possible to overwrite the defaults
% using explicit options in \includegraphics[width, height, ...]{}
\setkeys{Gin}{width=\maxwidth,height=\maxheight,keepaspectratio}
% Set default figure placement to htbp
\makeatletter
\def\fps@figure{htbp}
\makeatother
\setlength{\emergencystretch}{3em} % prevent overfull lines
\providecommand{\tightlist}{%
  \setlength{\itemsep}{0pt}\setlength{\parskip}{0pt}}
\setcounter{secnumdepth}{-\maxdimen} % remove section numbering
\usepackage{amsthm, mathtools}
\newtheorem{theorem}{Theorem}
\newtheorem{lemma}{Lemma}[theorem]
\usepackage{listings}
\def\lstlanguagefiles{lstlean.tex}
\lstset{language=lean}

\title{Poster Draft}
\author{Kexing Ying}
\date{}

\begin{document}
\maketitle

\hypertarget{the-stone-weierstrass-theorem}{%
\section{The Stone-Weierstrass
Theorem}\label{the-stone-weierstrass-theorem}}

\textit{The Stone-Weierstrass theorem} states that, given an unitial
subalgebra of \(M\), \(M_0\) that is closed under lattice operations and
separates points, \(\bar{M_0} = M\), and we say \(M_0\) is dense in
\(M\).

\section{Outline and Formalisation}
\textit{The Stone-Weierstrass theorem} was proved and formalised using the 
interactive theorem prover \textit{Lean} the source code of which can be found 
in my GitHub repository:\\ \url{http:\\github.com/JasonKYi/stone-weierstrass}. 

The theorem itself relies on two central lemmas:

{\bf Lemma 1.} For all $f \in M$, $f \in \bar{M_0}$ if and only if for all $x, y \in X$, $\epsilon > 0$, there 
exists $g \in M_0$ such that $\left| f(x) - g(x) \right| < \epsilon$ and $\left| f(y) - g(y) \right| < \epsilon$, i.e. 
there $M_0$ has a function arbitrarily close to $f$ at $x$ and $y$.

{\bf Lemma 2.} Given $S$, a subalgebra of $\mathbb{R}^2$, $S$ must be $\{(0,0)\}$, 
$\{(x, 0) \mid x \in \mathbb{R} \}$,
$\{(0. y) \mid y \in \mathbb{R} \}$, 
$\{(z, z) \mid z \in \mathbb{R} \}$, or
$\mathbb{R}^2$ itself.

\textit{Lemma 1} was formalised and is represented in Lean as 
$\tt{in\_closure_2\_iff\_dense\_at\_points}$ in $\tt{main.lean}$ the method 
of which we will discuss below. The forward direction of the proof is trivial so we will consider the reverse. 

Let us fix $x$ and $\epsilon$ and define a mapping to set of $X$
$$S : X \to \texttt{set} \hspace{1mm} X := \lambda y, \{z | f(z) - g_y(z) < \epsilon \},$$
where $g_y$ was chosen such that $\left| f(x) - g_y(x) \right| < \epsilon$ and $\left| f(y) - g_y(y) \right| < \epsilon$.

Then for all $y \in X$, $y \in S(y)$ so $\bigcup_{y\in X} S(y) = X$. But as $X$ is compact, $\bigcup_{y\in X} S(y)$ admits a finite subcover; 
so, there exists a finite index set $I$ such that $\bigcup_{i\in I} S(y_i) = X$. Thus, by letting $p_x = \bigvee_{i \in I} g_{y_i}$, we have constructed 
a function $p_x \in  \bar{M_0}$ such that 
$$
p_x(z) \ge g_{y_i}(z) > f(z) - \epsilon
$$
for all $z \in X$ and $i \in I$	.

Now, by defining a similar mapping to set of $X$, 
$$T : X \to \texttt{set} \hspace{1mm} X := \lambda x, \{z | p_x(z) < f(z) + \epsilon \},$$
we again create a finite subcover of $X$ and thus can create the required function with $\bigwedge_{j \in J}p_{x_j}$ where $J$ 
is the index set such that $\bigcup_{j\in J} T(p_{x_i}) = X$.

\textit{Lemma 2} was also formalised and is represented in Lean as $\tt{subalgebra\_of\_R2}$ and can be found in $\tt{ralgebra.lean}$.
The proof this lemma is rather tedious and follows directly by evoking the law of the excluded middle on different propositions 
multiple times.

Now, by considering \textit{lemma 1}, it can be deduced that, for $M_0$, $M_1$, closed subalgebras of $M$ under lattice 
operations and uniform convergence to the limit, $M_0 = M_1$ if and only if at for all distinct $x$, $y$, $M_0$ 
and $M_1$ have the same boundary points where the boundary points of $M_i$ at $x$, $y$ is defined to be $\{(f(x), f(y)) \mid f \in M_i \}$.
This was formalised in $\tt{eq\_iff\_boundary\_points\_eq}$ by constructing the notion of $\tt{closure'}$ in $\tt{definitions.lean}$

Lastly, as the boundary points of $M_0$ form a subalgebra of $\mathbb{R}^2$, we can utilise \textit{lemma 2} to 
deduce that the boundary points must either be $\{(z, z) \mid z \in \mathbb{R} \}$, or $\mathbb{R}^2$ (the first three possibilities 
in lemma 2 are not possible since $(1, 1)$ is in the boundary points). Now, if $M_0$ separates points then there must exist 
$f \in M_0$, $f(x) \neq f(y)$ so that excludes $\{(z, z) \mid z \in \mathbb{R} \}$ and hence the boundary points is $\mathbb{R}^2$ and
the theorem follows. This was formalised in $\tt{main.lean}$ with the statement being 

\begin{lstlisting}
theorem weierstrass_stone \{M₀' : subalgebra ℝ (X → ℝ)\} 
(hc   : closure₀ M₀'.carrier = M₀'.carrier)
(hsep : has_seperate_points M₀'.carrier) :
closure₂ M₀'.carrier = univ
\end{lstlisting}


\hypertarget{trigonometric-polynomials}{%
\section{Trigonometric Polynomials}\label{trigonometric-polynomials}}

Similarly to normal polynomials, we can deduce that trigonometric
polynomials are dense in bounded continuous functions on \([0, 2\pi]\)
where trigonometric polynomials are functions of the form
\[f(x) = a_0 + \sum_{i = 1}^n a_n \cos(nx) + b_n \sin(nx)\] By
considering the identities of multiplication between trigonometric
functions, we can easily see that the trigonometric polynomials form a
unitial subalgebra that seperates points, and therefore dense by
Stone-Weierstrass.

\end{document}
