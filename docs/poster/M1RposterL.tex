\documentclass[landscape,final,a0paper]{baposter}


\tracingstats=2

\usepackage{calc}
\usepackage{graphicx}
\usepackage{amsmath}
\usepackage{amssymb}
\usepackage{relsize}
\usepackage{multirow}
\usepackage{bm}

\usepackage{graphicx}
\usepackage{multicol}

\usepackage{pgfbaselayers}
\pgfdeclarelayer{background}
\pgfdeclarelayer{foreground}
\pgfsetlayers{background,main,foreground}

\usepackage{times}
\usepackage{helvet}
\usepackage{palatino}
\usepackage{hyperref}
\usepackage[utf8x]{inputenc}
\usepackage{amssymb, upgreek}

\usepackage{color}
\definecolor{keywordcolor}{rgb}{0.7, 0.1, 0.1}   % red
\definecolor{commentcolor}{rgb}{0.4, 0.4, 0.4}   % grey
\definecolor{symbolcolor}{rgb}{0.0, 0.1, 0.6}    % blue
\definecolor{sortcolor}{rgb}{0.1, 0.5, 0.1}      % green
\definecolor{errorcolor}{rgb}{1, 0, 0}           % bright red
\definecolor{stringcolor}{rgb}{0.5, 0.3, 0.2}    % brown

\usepackage{listings}
\def\lstlanguagefiles{lstlean.tex}
\lstset{language=lean}

\hypersetup{
    colorlinks=true,
    linkcolor=blue,
    filecolor=magenta,      
    urlcolor=cyan,
}

\urlstyle{same}
\DeclareUrlCommand\url{\color{magenta}\def\UrlLeft{http://}\urlstyle{tt}}

\newcommand{\captionfont}{\footnotesize}

\selectcolormodel{cmyk}

%\graphicspath{{images/}}

%%%%%%%%%%%%%%%%%%%%%%%%%%%%%%%%%%%%%%%%%%%%%%%%%%%%%%%%%%%%%%%%%%%%%%%%%%%%%%%%
%%%% Some math symbols used in the text
%%%%%%%%%%%%%%%%%%%%%%%%%%%%%%%%%%%%%%%%%%%%%%%%%%%%%%%%%%%%%%%%%%%%%%%%%%%%%%%%
% Format 

\renewcommand{\Pr}{\mbox{P}}
\newcommand{\e}{\mbox{e}}
\newcommand{\dx}{\,\mbox{d}x}

%%%%%%%%%%%%%%%%%%%%%%%%%%%%%%%%%%%%%%%%%%%%%%%%%%%%%%%%%%%%%%%%%%%%%%%%%%%%%%%%
% Multicol Settings
%%%%%%%%%%%%%%%%%%%%%%%%%%%%%%%%%%%%%%%%%%%%%%%%%%%%%%%%%%%%%%%%%%%%%%%%%%%%%%%%
\setlength{\columnsep}{0.7em}
\setlength{\columnseprule}{0mm}


%%%%%%%%%%%%%%%%%%%%%%%%%%%%%%%%%%%%%%%%%%%%%%%%%%%%%%%%%%%%%%%%%%%%%%%%%%%%%%%%
% Save space in lists. Use this after the opening of the list
%%%%%%%%%%%%%%%%%%%%%%%%%%%%%%%%%%%%%%%%%%%%%%%%%%%%%%%%%%%%%%%%%%%%%%%%%%%%%%%%
\newcommand{\compresslist}{%
\setlength{\itemsep}{1pt}%
\setlength{\parskip}{0pt}%
\setlength{\parsep}{0pt}%
}


%%%%%%%%%%%%%%%%%%%%%%%%%%%%%%%%%%%%%%%%%%%%%%%%%%%%%%%%%%%%%%%%%%%%%%%%%%%%%%
%%% Begin of Document
%%%%%%%%%%%%%%%%%%%%%%%%%%%%%%%%%%%%%%%%%%%%%%%%%%%%%%%%%%%%%%%%%%%%%%%%%%%%%%

\begin{document}

%%%%%%%%%%%%%%%%%%%%%%%%%%%%%%%%%%%%%%%%%%%%%%%%%%%%%%%%%%%%%%%%%%%%%%%%%%%%%%
%%% Here starts the poster
%%%---------------------------------------------------------------------------
%%% Format it to your taste with the options
%%%%%%%%%%%%%%%%%%%%%%%%%%%%%%%%%%%%%%%%%%%%%%%%%%%%%%%%%%%%%%%%%%%%%%%%%%%%%%
% Define some colors
\definecolor{silver}{cmyk}{0,0,0,0.3}
\definecolor{yellow}{cmyk}{0,0,0.9,0.0}
\definecolor{reddishyellow}{cmyk}{0,0.22,1.0,0.0}
\definecolor{black}{cmyk}{0,0,0.0,1.0}
\definecolor{darkYellow}{cmyk}{0,0,1.0,0.5}
\definecolor{darkSilver}{cmyk}{0,0,0,0.1}
\definecolor{lightyellow}{cmyk}{0,0,0.3,0.0}
\definecolor{lighteryellow}{cmyk}{0,0,0.1,0.0}
\definecolor{lighteryellow}{cmyk}{0,0,0.1,0.0}
\definecolor{lightestyellow}{cmyk}{0,0,0.05,0.0}
\definecolor{cyan}{cmyk}{1,0,0,0}
\definecolor{lightcyan}{cmyk}{0.5,0,0,0}
\definecolor{pastelcyan}{cmyk}{0.25,0,0,0}
\definecolor{magenta}{cmyk}{0,1,0,0}
\definecolor{yellow}{cmyk}{0,0,1,0}
\definecolor{lightyellow}{cmyk}{0,0,0.5,0}
\definecolor{pastelyellow}{cmyk}{0,0,0.25,0}
\definecolor{black}{cmyk}{0,0,0,1}
\definecolor{darkgray}{cmyk}{0,0,0,0.75}
\definecolor{gray}{cmyk}{0,0,0,0.5}
\definecolor{lightgray}{cmyk}{0,0,0,0.25}
\definecolor{white}{cmyk}{0,0,0,0}
\definecolor{red}{cmyk}{0,1,1,0}
\definecolor{orange}{cmyk}{0,0.5,1,0}
\definecolor{scarlet}{cmyk}{0,1,0.5,0}
\definecolor{brown}{cmyk}{0.5,0.75,1,0}
\definecolor{camel}{cmyk}{0.25,0.375,0.5,0}
\definecolor{cream}{cmyk}{0,0.2,0.3,0}
\definecolor{green}{cmyk}{1,0,1,0}
\definecolor{lightgreen}{cmyk}{0.5,0,0.5,0}
\definecolor{pastelgreen}{cmyk}{0.25,0,0.25,0}
\definecolor{mossgreen}{cmyk}{0.64,0.4,1,0}
\definecolor{yellowgreen}{cmyk}{0.5,0,1,0}
\definecolor{skyblue}{cmyk}{0.4,0.16,0,0}
\definecolor{royal}{cmyk}{1.0,0.5,0,0}
\definecolor{navyblue}{cmyk}{0.9,0.75,0.5,0}
\definecolor{lightnavy}{cmyk}{0.4,0.3,0.2,0}
\definecolor{blue}{cmyk}{1,1,0,0}
\definecolor{lightblue}{cmyk}{0.5,0.5,0,0}
\definecolor{pastelblue}{cmyk}{0.25,0.25,0,0}
\definecolor{lightpastelblue}{cmyk}{0.15,0.15,0,0}
\definecolor{lightestpastelblue}{cmyk}{0.05,0.05,0,0}
\definecolor{lavender}{cmyk}{0.25,0.25,0,0}
\definecolor{violet}{cmyk}{0.75,1,0.25,0}
\definecolor{purple}{cmyk}{0.5,1,0.5,0}
\definecolor{lightpurple}{cmyk}{0.25,0.5,0.25,0}
\definecolor{pink}{cmyk}{0,0.5,0,0}


%%

\typeout{Poster Starts}
%\background{
  %\begin{tikzpicture}[remember picture,overlay]%
  %  \draw (current page.north west)+(-2em,-2em) node[anchor=north west] %{\hspace{-2em}\includegraphics[height=1.1\textheight]{silhouettes_background}};
 % \end{tikzpicture}%
%}




\newlength{\leftimgwidth}
\begin{poster}%
  % Poster Options, such as colours etc
  {
  % Show grid to help with alignment
  grid=false,
 % Column spacing
  colspacing=0.5em,
 % Color style
 % bgColorOne=pastelblue,
 %bgColorTwo=lightpastelblue,
  bgColorOne=white,
  bgColorTwo=white,
  borderColor=navyblue,
  headerColorOne=lightnavy,
  headerColorTwo=purple,
  headerFontColor=black,
 % boxColorOne=lightpastelblue,
 % boxColorTwo=lightestpastelblue,
 boxColorOne=white,
 boxColorTwo=white,
 % Format of textbox
  textborder=roundedleft,
% textborder=rectangle,
% Format of text header
  eyecatcher=false,
  headerborder=open,
  headerheight=0.08\textheight,
  headershape=roundedright,
  headershade=plain,
  headerfont=\Large\textsf, %Sans Serif
  boxshade=plain,
%  background=shade-tb,
 % background=plain,
  background=none,
  linewidth=2pt
  }
  % Eye Catcher
  {} % select eyecatcher=false above if not required. If no eye catcher is present, the title is left aligned.
  % Title
  {\sf %Sans Serif
  %\bf% Serif
  The Stone-Weierstrass Theorem and its Formalisation}
  % Authors
  {
  Kexing Ying
  }
  % University logo
  { % The makebox allows the title to flow into the logo
    \makebox[8em][r]{%
        \begin{minipage}{16em}
				\hfill \includegraphics[height=3em]{imperial.pdf}
				\end{minipage}
      
    }
  }

  \tikzstyle{light shaded}=[top color=baposterBGtwo!30!white,bottom color=baposterBGone!30!white,shading=axis,shading angle=30]

  % Width of left inset image
     \setlength{\leftimgwidth}{0.78em+8.0em}

%%%%%%%%%%%%%%%%%%%%%%%%%%%%%%%%%%%%%%%%%%%%%%%%%%%%%%%%%%%%%%%%%%%%%%%%%%%%%%
%%% Now define the boxes that make up the poster
%%%---------------------------------------------------------------------------
%%% Each box has a name and can be placed absolutely or relatively.
%%% The only inconvenience is that you can only specify a relative position 
%%% towards an already declared box. So if you have a box attached to the 
%%% bottom, one to the top and a third one which should be in between, you 
%%% have to specify the top and bottom boxes before you specify the middle 
%%% box.
%%%%%%%%%%%%%%%%%%%%%%%%%%%%%%%%%%%%%%%%%%%%%%%%%%%%%%%%%%%%%%%%%%%%%%%%%%%%%%
    %
    % A coloured circle useful as a bullet with an adjustably strong filling
\newcommand{\colouredcircle}[1]{%
      \tikz{\useasboundingbox (-0.2em,-0.32em) rectangle(0.2em,0.32em); \draw[draw=black,fill=baposterBGone!80!black!#1!white,line width=0.03em] (0,0) circle(0.18em);}}


%%%%%%%%%%%%%%%%%%%%%%%%%%%%%%%%%%%%%%%%%%%%%%%%%%%%%%%%%%%%%%%%%%%%%%%%%%%%%%
  \headerbox{Notations and Variables}{name=notations,column=0,row = 0}{
%%%%%%%%%%%%%%%%%%%%%%%%%%%%%%%%%%%%%%%%%%%%%%%%%%%%%%%%%%%%%%%%%%%%%%%%%%%%%%
Throughout this poster, we will let $X$ be a compact metric space, $M$ to be the set of all 
bounded and continuous functions from $X$ to $\mathbb{R}$, $M_0$ a subset of $M$, 
$\bar{M_0}$ the closure of $M_0$ under lattice operations and uniform convergence to 
the limit and unless otherwise specified, we let all functions from $X$ to $\mathbb{R}$ 
be bounded and continuous.

\vspace{3mm}
We say $M_0$ \textit{separates points} if and only if for all distinct $x, y \in X$, there 
exists some $f \in M_0$ such that $f(x) \neq f(y)$.

  \vspace{0.3em}
  }

%%%%%%%%%%%%%%%%%%%%%%%%%%%%%%%%%%%%%%%%%%%%%%%%%%%%%%%%%%%%%%%%%%%%%%%%%%%%%%
  \headerbox{Lattice Operations}{name=lattice,column=0,below=notations}{
%%%%%%%%%%%%%%%%%%%%%%%%%%%%%%%%%%%%%%%%%%%%%%%%%%%%%%%%%%%%%%%%%%%%%%%%%%%%%%
While we do not form a complete lattice on the set of bounded continuous functions, 
we define two lattice operations \(\vee, \wedge : (X \to \mathbb{R})^2\to (X \to \mathbb{R}) \) 
such that, for all \(f, g : X \to \mathbb{R}\), where \(f, g\) are bounded continuous functions 
$$
f \vee g = \max\{f, g\},
$$
and 
$$
f \wedge g = \min \{f, g\}.
$$

  \vspace{0.3em}
  }

%%%%%%%%%%%%%%%%%%%%%%%%%%%%%%%%%%%%%%%%%%%%%%%%%%%%%%%%%%%%%%%%%%%%%%%%%%%%%%
  \headerbox{$R$-algebra and Subalgebra}{name=algebra,column=0,below=lattice}{
%%%%%%%%%%%%%%%%%%%%%%%%%%%%%%%%%%%%%%%%%%%%%%%%%%%%%%%%%%%%%%%%%%%%%%%%%%%%%%
While in many literature (cite) algebras over a field are used to examine the 
Stone-Weierstrass theorem, it is in fact not necessary to examine algebras over field. 
As we will demonstrate, algebras over a commutative ring (or $R$-algebras) will suffice.

\vspace{3mm}
An \textit{$R$-algebra} is a mathematical object consisting of a commutative ring $R$ and a 
semi-ring $A$ such that there exists scalar multiplication \(\cdot : R \times A \to A\) 
and a homomorphism from $R$ to $A$, \(\phi : R \to A\) such that 
$$
r \cdot a = \phi(r) \times a,
$$
and 
$$
\phi(r) \times a = a \times \phi(r)
$$
are satisfied for all $r \in R$ and $a \in A$ (cite).

\vspace{3mm}
A \textit{subalgebra} $S$ of an $R$-algebra $A$ is a subset of $A$ that's closed under the induced operations 
carried from $A$. 

\vspace{3mm}
It can be shown that both $M$ and $\mathbb{R}^2$ form a $R$-algebra over $\mathbb{R}$ (cite my own proof?).

  \vspace{0.3em}
  }

%%%%%%%%%%%%%%%%%%%%%%%%%%%%%%%%%%%%%%%%%%%%%%%%%%%%%%%%%%%%%%%%%%%%%%%%%%%%%%
 

%%%%%%%%%%%%%%%%%%%%%%%%%%%%%%%%%%%%%%%%%%%%%%%%%%%%%%%%%%%%%%%%%%%%%%%%%%%%%%
  \headerbox{The Stone-Weierstrass Theorem}{name=stone,column=1,span=1, row=0}{
\textit{The Stone-Weierstrass theorem} states that, given a subalgebra of $M$, $M_0$ that is 
closed under lattice operations and separates points, $\bar{M_0} = M$, and we say $M_0$ is dense in $M$.

 }


%%%%%%%%%%%%%%%%%%%%%%%%%%%%%%%%%%%%%%%%%%%%%%%%%%%%%%%%%%%%%%%%%%%%%%%%%%%%%%
\headerbox{Outline of the proof}{name=outline,column=1,span=1,below=stone}{
The theorem relies on two crucial lemmas:

\vspace{2mm}
{\bf Lemma 1.} For all $f \in M$, $f \in \bar{M_0}$ if and only if for all $x, y \in X$, $\epsilon > 0$, there 
exists $g \in M_0$ such that $\left| f(x) - g(x) \right| < \epsilon$ and $\left| f(y) - g(y) \right| < \epsilon$, i.e. 
there $M_0$ has a function arbitrarily close to $f$ at $x$ and $y$.

\vspace{2mm}
{\bf Lemma 2.} Given $S$, a subalgebra of $\mathbb{R}^2$, $S$ must be $\{(0,0)\}$, 
$\{(x, 0) \mid x \in \mathbb{R} \}$,
$\{(0. y) \mid y \in \mathbb{R} \}$, 
$\{(z, z) \mid z \in \mathbb{R} \}$, or
$\mathbb{R}^2$ itself.

\vspace{3mm}
By considering \textit{lemma 1}, it can be deduced that, for $M_0$, $M_1$, closed subalgebras of $M$ under lattice 
operations and uniform convergence to the limit, $M_0 = M_1$ if and only if at for all distinct $x$, $y$, $M_0$ 
and $M_1$ have the same boundary points (the boundary points of $M_i$ at $x$ $y$ is defined to be $\{(f(x), f(y)) \mid f \in M_i \}$).

\vspace{2mm}
Now by considering the fact that the boundary points of $M_0$ form a subalgebra of $\mathbb{R}^2$, we can utilise \textit{lemma 2} to 
deduce that the boundary points must either be $\{(z, z) \mid z \in \mathbb{R} \}$, or $\mathbb{R}^2$ (the first three possibilities 
in lemma 2 are not possible since $(1, 1)$ is in the boundary points). Now, if $M_0$ separates points then there must exist 
$f \in M_0$, $f(x) \neq f(y)$ so that excludes $\{(z, z) \mid z \in \mathbb{R} \}$ and hence the boundary points is $\mathbb{R}^2$ and
the theorem follows.

  \vspace{0.5em}
  }
  
 %%%%%%%%%%%%%%%%%%%%%%%%%%%%%%%%%%%%%%%%%%%%%%%%%%%%%%%%%%%%%%%%%%%%%%%%%%%%%
\headerbox{Formalisation}{name=form,column=2,span=2,row=0}{

\begin{minipage}[t]{0.475\textwidth}
The procedure in which the formalisation was achieved is similar to the outline, the source code of which can be found in my GitHub repository:\\
\url{github.com/JasonKYi/weierstrass_approximation}

\vspace{2mm}
\textit{Lemma 1} was formalised and is represented in Lean as 
$\tt{in\_closure₂\_iff\_dense\_at\_points}$ in $\tt{main.lean}$ the method 
of which we will discuss below. The forward direction of the proof is trivial so let us consider the reverse. 

\vspace{2mm}
Let us fix $x$ and $\epsilon$ and define a mapping to set of $X$
$$S : X \to \texttt{set} \hspace{1mm} X := \lambda y, \{z | f(z) - g_y(z) < \epsilon \},$$
where $g_y$ was chosen such that $\left| f(x) - g_y(x) \right| < \epsilon$ and $\left| f(y) - g_y(y) \right| < \epsilon$.

Then for all $y \in X$, $y \in S(y)$ so $\bigcup_{y\in X} S(y) = X$. But as $X$ is compact, $\bigcup_{y\in X} S(y)$ admits a finite subcover (cite); 
so, there exists a finite index set $I$ such that $\bigcup_{i\in I} S(y_i) = X$. Thus, by letting $p_x = \bigvee_{i \in I} g_{y_i}$, we have constructed 
a function $p_x \in  \bar{M_0}$ such that 
$$
p_x(z) \ge g_{y_i}(z) > f(z) - \epsilon
$$
for all $z \in X$ and $i \in I$	.

Now, by defining a similar mapping to set of $X$, 
$$T : X \to \texttt{set} \hspace{1mm} X := \lambda x, \{z | p_x(z) < f(z) + \epsilon \},$$
we again create a finite subcover of $X$ and thus can create the required function with $\bigwedge_{j \in J}p_{x_j}$ where $J$ 
is the index set such that $\bigcup_{j\in J} T(p_{x_i}) = X$.
\end{minipage}
\begin{minipage}[t]{0.05\textwidth}
\hspace{0mm}
\end{minipage}
\begin{minipage}[t]{0.46\textwidth}

\textit{Lemma 2} was also formalised and is represented in Lean as $\tt{subalgebra\_of\_R2}$ and can be found in $\tt{ralgebra.lean}$.
The proof this lemma is rather tedious and follows directly by evoking the law of the excluded middle on different propositions 
multiple times.

\vspace{3mm}
Thus, with both lemma 1 and lemma 2 in our arsenal, it was shown that given two subalgebra of $M$, $M_0$ and $M_1$ which are 
closed under lattice operations, $M_0 = M_1$ if and only if they have the same boundary points with $\tt{eq\_iff\_boundary\_points\_eq}$ 
by constructing the notion of $\tt{closure'}$ in $\tt{definitions.lean}$ for sets of points in $\mathbb{R}^2$.

\vspace{3mm}
With that, by defining the notion of $\tt{has\_seperate\_points}$ in $\tt{definitions.lean}$ the Stone-Weierstrass theorem was proved with 
the method described in the outline with theorem statement in Lean being \\ 

\lstinline{theorem weierstrass_stone \{M₀' : subalgebra ℝ (X → ℝ)\} 
(hc   : closure₀ M₀'.carrier = M₀'.carrier)
(hsep : has_seperate_points M₀'.carrier) :
closure₂ M₀'.carrier = univ}

\vspace{3mm}
with the $\tt{M₀'.carrier}$ notation referring to the underlying subset of $\tt{M₀'}$ the subalgebra and $\tt{hc}$ indicating $\tt{M₀'.carrier}$ 
is closed under lattice operations.

\end{minipage}

\vspace{0.2em}
  }

%%%%%%%%%%%%%%%%%%%%%%%%%%%%%%%%%%%%%%%%%%%%%%%%%%%%%%%%%%%%%%%%%%%%%%%%%%%%%%
  \headerbox{Approximation Theorem}{name=approx,column=1,span=2,below=form}{

  \vspace{0.2em}
  }


%%%%%%%%%%%%%%%%%%%%%%%%%%%%%%%%%%%%%%%%%%%%%%%%%%%%%%%%%%%%%%%%%%%%%%%%%%%%%%

%%%%%%%%%%%%%%%%%%%%%%%%%%%%%%%%%%%%%%%%%%%%%%%%%%%%%%%%%%%%%%%%%%%%%%%%%%%%%%
\headerbox{Conclusions}{name=concl,column=3,below=form}{

	}
	
%%%%%%%%%%%%%%%%%%%%%%%%%%%%%%%%%%%%%%%%%%%%%%%%%%%%%%%%%%%%%%%%%%%%%%%%%%%%%%
  \headerbox{References}{name=references,column=3,above=bottom}{
    \smaller
    \vspace{-0.4em}
    \bibliographystyle{plain}
    \renewcommand{\section}[2]{\vskip 0.05em}
      \begin{thebibliography}{1}\itemsep=-0.01em
      \setlength{\baselineskip}{0.4em}
      \bibitem{stir}
        D. Stirzaker.
        \newblock Elementary Probability. 
        \newblock Cambridge University Press, 2003.
      \bibitem{ross}
        S. Ross.
        \newblock A First Course in Probability.
        \newblock Prentice Hall, 2001.
				\bibitem{grimmett}
				G. Grimmett. 
				\newblock Probability and Random Processes.
				\newblock Oxford University Press, 2001.
      \end{thebibliography}
  }
%%%%%%%%%%%%%%%%%%%%%%%%%%%%%%%%%%%%%%%%%%%%%%%%%%%%%%%%%%%%%%%%%%%%%%%%%%%%%%
  
\end{poster}

\end{document}
